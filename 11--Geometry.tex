% LaTeX Algebra Functions by Larry A. Hartman 
% is licensed under a Creative Commons 
% Attribution-NonCommercial 4.0 International License.
%
% To view a copy of this license, visit 
% http://creativecommons.org/licenses/by-nc/4.0/.
%
% Permissions beyond the scope of this license may be 
% available at https://wavesofvoqueric.com/portfolio/latex/.




% DESCRIPTION of 11--Geometry.tex
% ==========
%
% This file contains commands that are primarily described as 
% algebraic in nature.
%
% The latest copy can be obtained from the following website:
%
% https://github.com/voqueric/LaTex-Math-and-Science-Functions
%
% Every attempt will be made to keep command names and arguements
% the same in successive versions to maintain backward compatibility.




% INSTALLATION
% ==========
% This document is automatically loaded by the higher-level document
% 00--Science_and_Math.tex, and should not be loaded external to this document.
%
% Use the following line within the document preamble to determine if compilation
% take place in a Linux operating system or Windows operating system:
%
% \InputIfFileExists{/linux/path/to/file/00--Science_and_Math.tex}%
%   {}%
%   {\input{C:/windows/path/to/file/00--Science_and_Math.tex}}
%
% This file must be loaded prior to loading other files contained in the
% https://github.com/voqueric/LaTex-Math-and-Science-Functions repository.
 
 


% GENERAL NOTES
% ==========
% - Commands are grouped according to type.
% - Each command is provided with:
%       Command name
%       Description
%       Example or Equation (visual approximation using text format)
%       Arguments
% - Commands representing equations type-set only the right-side 
%   of the provided equation




% COMMAND NAME CONVENTIONS
% ==========
% - First letter represents typesetting mode: 
%       -- Math form (prefixed with 'm')
%       -- Inline form (prefixed with 'i')
%       -- Inline forms that do not differ from counterpart math forms
%          are simply mapped to the math forms.
% - Second letter represents the major subdiscipline of the command:
%       -- Algebra ('a')
%       -- Calculus ('c')
%       -- Metrology ('m')
%       -- Trigonometry ('t')
%       -- Physics ('p')
% - Third letters and beyond represent abbreviated forms of the command functions.




% mgptri/igptri
% Math - Geom - Perimeter of Triangle
%
%   P = a + b + c
%
%   arg a: side a
%   arg b: side b
%   arg c: side c
\newcommand{ \mgptri }[3]{ #1 + #2 + #3 }
\newcommand{ \igptri }[3]{ \aptri{ #1 }{ #2 }{ #3 } }


% mghftrif/ighftrif
% Math - Geom - Herons Formula Full
%
%         P   P        P        P
%   A = [ - ( - - a )( - - b )( - - c ) ]^(1/2)
%         2   2        2        2
%
%   arg P: perimeter
%   arg a: side a
%   arg b: side b
%   arg c: side c
\newcommand{ \mghftrif }[4]{ \Bigg[ \dfrac{ { #1 }_{ #2 #3 #4 } }{ 2 } \bigg( \dfrac{ { #1 }_{ #2 #3 #4 } }{ 2 } - #2 \bigg) \bigg( \dfrac{ { #1 }_{ #2 #3 #4 } }{ 2 } - #3 \bigg) \bigg( \dfrac{ { #1 }_{ #2 #3 #4 } }{ 2 } - #3 \bigg) \Bigg]^{ \sfrac{ 1 }{ 2 } } }
\newcommand{ \ighftrif }[4]{ [ \frac{ { #1 }_{ #2 #3 #4 } }{ 2 } ( \frac{ { #1 }_{ #2 #3 #4 } }{ 2 } - #2 ) ( \frac{ { #1 }_{ #2 #3 #4 } }{ 2 } - #3 ) ( \frac{ { #1 }_{ #2 #3 #4 } }{ 2 } - #3 ) ]^{ \sfrac{ 1 }{ 2 } } }


% mghftric/ighftric
% Math - Geom - Herons Formula Compact
%
%   A = [ S ( S - side1 )( S - side2 )( S - side3 ) ]^(1/2): S = P_abc / 2
%
%   arg S: perimeter / 2
%   arg P: perimeter
%   arg a: side a
%   arg b: side b
%   arg c: side c
\newcommand{ \mghftric }[5]{ \big[ #1 ( #1 - #3 )( #1 - #4 )( #1 - #5 ) \big]^{ \sfrac{ 1 }{ 2 } }: \qquad #1 = \dfrac{ { #2 }_{ #3 #4 #5 } }{ 2 } }
\newcommand{ \ighftric }[5]{ [ #1 ( #1 - #3 )( #1 - #4 )( #1 - #5 ) ]^{ \sfrac{ 1 }{ 2 } }: \qquad #1 = \sfrac{ { #2 }_{ #3 #4 #5 } }{ 2 } }


% mgartri/igartri
% Math - Geom - Area of Right-Triangle
%
%       1
%   A = - bh
%       2
%
%       1
%   A = - hb
%       2
%
%   arg b: base      alt-arg h: height
%   arg h: height    alt-arg b: base
\newcommand{ \mgartri }[2]{ \dfrac{ 1 }{ 2 }{ #1 }{ #2 } }
\newcommand{ \igartri }[2]{ \frac{ 1 }{ 2 }{ #1 }{ #2 } }


% mgpts/igpts
% Math - Geom - Pythagorean Theorem - Squared Form Sum
%
%   h^2 = a^2 + o^2
%
%   arg a: adjacent side
%   arg o: opposite side
\newcommand{ \mgpts }[2]{ { #1 }^2 + { #2 }^2 }
\newcommand{ \igpts }[2]{ \mapts{ #1 }{ #2 } }


% mgptsd/igptsd
% Math - Geom - Pythagorean Theorem - Squared Form Difference
%
%   a^2 = h^2 - o^2
%
%   arg h: hypotenuse
%   arg o: opposite side
\newcommand{ \mgptsd }[2]{ { #1 }^2 - { #2 }^2 }
\newcommand{ \igptsd }[2]{ \maptsd{ #1 }{ #2 } }


% mgptr/igptr
% Math - Geom - Pythagorean Theorem - Root Form Sum
%
%   h = (a^2 + o^2)^(1/2)
%
%   arg a: adjacent side
%   arg o: opposite side
\newcommand{ \mgptr }[2]{ ( { #1 }^2 + { #2 }^2 )^{ \sfrac{ 1 }{ 2 } } }
\newcommand{ \igptr }[2]{ \maptr{ #1 }{ #2 } }


% mgptrd/igptrd
% Math - Geom - Pythagorean Theorem - Root Form Difference
%
%   a = (h^2 - o^2)^(1/2)
%
%   arg h: hypotenuse
%   arg o: opposite side
\newcommand{ \mgptrd }[2]{ ( { #1 }^2 - { #2 }^2 )^{ \sfrac{ 1 }{ 2 } } }
\newcommand{ \igptrd }[2]{ \maptrd{ #1 }{ #2 } }