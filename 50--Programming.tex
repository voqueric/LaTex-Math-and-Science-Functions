% LaTeX Math and Science Functions by Larry A. Hartman 
% is licensed under a Creative Commons 
% Attribution-NonCommercial 4.0 International License.
%
% To view a copy of this license, visit 
% http://creativecommons.org/licenses/by-nc/4.0/.
%
% Permissions beyond the scope of this license may be 
% available at https://wavesofvoqueric.com/portfolio/latex/.




% DESCRIPTION of 50--Programming.tex
% ==========
%
% This file contains commands that are primarily described as 
% algebraic in nature.
%
% The latest copy can be obtained from the following website:
%
% https://github.com/voqueric/LaTex-Math-and-Science-Functions
%
% Every attempt will be made to keep command names and arguements
% the same in successive versions to maintain backward compatibility.




% INSTALLATION
% ==========
% This document is automatically loaded by the higher-level document
% 00--Science_and_Math.tex, and should not be loaded external to this document.
%
% Use the following line within the document preamble to determine if compilation
% take place in a Linux operating system or Windows operating system:
%
% \InputIfFileExists{/linux/path/to/file/00--Science_and_Math.tex}%
%   {}%
%   {\input{C:/windows/path/to/file/00--Science_and_Math.tex}}
%
% This file must be loaded prior to loading other files contained in the
% https://github.com/voqueric/LaTex-Math-and-Science-Functions repository.
 
 


% GENERAL NOTES
% ==========
% - Commands are grouped according to type.
% - Each command is provided with:
%       Command name
%       Description
%       Example or Equation (visual approximation using text format)
%       Arguments
% - Commands representing equations type-set only the right-side 
%   of the provided equation




% COMMAND NAME CONVENTIONS
% ==========
% - First letter represents typesetting mode: 
%       -- Math form (prefixed with 'm')
%       -- Inline form (prefixed with 'i')
%       -- Inline forms that do not differ from counterpart math forms
%          are simply mapped to the math forms.
% - Second letter represents the major subdiscipline of the command:
%       -- Algebra ('a')
%       -- Calculus ('c')
%       -- Metrology ('m')
%       -- Trigonometry ('t')
%       -- Physics ('p')
% - Third letters and beyond represent abbreviated forms of the command functions.




% Colors for programming language environments
\definecolor{matlabgreen}{rgb}{0,0.6,0}
\definecolor{matlabgray}{rgb}{0.5,0.5,0.5}
\definecolor{matlabpurple}{rgb}{0.58,0,0.82}
\definecolor{matlabbackground}{rgb}{0.95,0.95,0.92}




% Matlab environment style
\lstdefinestyle{stymatlab}{
    backgroundcolor=\color{matlabbackground},   
    commentstyle=\color{matlabgreen},
    keywordstyle=\color{magenta},
    numberstyle=\tiny\color{matlabgray},
    stringstyle=\color{matlabpurple},
    basicstyle=\ttfamily\footnotesize,
    breakatwhitespace=false,         
    breaklines=true,                 
    captionpos=b,                    
    keepspaces=true,                 
    numbers=left,                    
    numbersep=5pt,                  
    showspaces=false,                
    showstringspaces=false,
    showtabs=false,                  
    tabsize=4
}

